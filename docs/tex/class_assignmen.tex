\documentclass{extbook}
\usepackage[utf8]{inputenc}
\usepackage{amsmath}
\usepackage{fancyhdr}
\usepackage{geometry}

\geometry{
	top=50mm,
	right=50mm,
	left=50mm
}

\newcommand{\resetWhitespace}{
	\setlength{\abovedisplayskip}{4pt}
	\setlength{\belowdisplayskip}{4pt}
}

\pagestyle{fancyplain}
\fancyhf{}
\fancyhead[C]{\small\textit{Theory of information}}
\fancyhead[L]{\small{292}}
\renewcommand{\headrulewidth}{0pt}

\setlength{\headsep}{-10pt}

\begin{document}

\begin{center}

\end{center}

It is possible to break a choice of one event from the set $\{E_{1}, E_{2}, \dots,$ $E_{n}\}$ into two steps: at first, to choose a group where the chosen event is and then to choose the event from the already chosen group. By Axiom A3, this gives us the following equality
\begin{align*}
	\label{eq:3.5.10}
	\resetWhitespace
	H(1/n, 1/n, \dots , 1/n) = H(p_{1}, p_{2}, \dots ,p_{n}) + \sum\nolimits_{i=1}^{k}p_{i}H(1/n_{i}, 1/n_{i}, \dots ,
	1/n_{i})                                                                                                           \\
	\tag{3.5.10}
\end{align*}

It is proved that
\begin{equation*}
	\resetWhitespace
	H(1/n, 1/n, \dots , 1/n) = \log_{2}n
\end{equation*}

\noindent and thus,
\begin{equation*}
	\resetWhitespace
	H(1/n_{i}, 1/n_{i}, \dots , 1/n_{i}) = \log_{2}n_{i}
\end{equation*}

\noindent for all $i=1, 2, 3, \dots , n$.

This and the equality (\ref{eq:3.5.10}), give us
\begin{equation}
	\resetWhitespace
	\log_{2}n = H(p_{1}, p_{2}, \dots , p_{n}) + \sum\nolimits_{i=1}^{k}p_{i}\log_{2}n_{i}
	\tag{3.5.11}
\end{equation}

Consequently, we have
\setlength{\abovedisplayskip}{0pt}
\begin{flalign*}
	\resetWhitespace
	H&(p_{1}, p_{2}, \dots , p_{n}) = \log_{2}n - \sum\nolimits_{i=1}^{k}p_{i}\log_{2}n_{i}\\
	& = - \sum\nolimits_{i=1}^{k}p_{i}log_{2}n_{i}/n = - \sum\nolimits_{i=1}^{k}p_{i}\log_{2}p_{i}
\end{flalign*}

This proves formula (3.2.4) for all rational numbers. As by Axiom A1, $H(p_{i}, p_{2}, \dots , p_{n})$ is a continuous function and the set of all rational numbers is dense in the set of all real numbers, we have
\begin{equation*}
	\resetWhitespace
	H(p_{i}, p_{2}, \dots , p_{n}) = - \sum\nolimits_{i=1}^{k}p_{i} \log_{2}p_{i}
\end{equation*}

\noindent for all real numbers.

Theorem is proved.

Of we take Axiom A4 away, we have a weaker result.

\textbf{Theorem 3.5.2} (Shannon, 1948). The only function that satisfies Axioms A1-A3 and A5 is the information entropy $H(p_{i}, p_{2}, \dots , p_{n})$ determined by the formula
\begin{equation}
	\resetWhitespace
	H(p_{i}, p_{2}, \dots , p_{n}) = K\sum\nolimits_{i=1}^{k}p_{i} \log_{2}p_{i}
	\tag{3.5.12}
\end{equation}

\noindent where K is an arbitrary constant.

\end{document}
